\documentclass{article}
%\usepackage{graphics}
%\usepackage{hyperref}
\usepackage{amsmath}
\usepackage{graphicx}
\usepackage[utf8]{inputenc}
\usepackage{color}

%opening
\title{Numerical SSL solution of melt polymer brush }
\author{Jiuzhou Tang}

\begin{document}

\maketitle

\begin{abstract}

\end{abstract}

\section{SSL theory of melt polymer brush}
The most import characteteristic of the polymer brush is that one end of the chain is grafted onto a solid surface at z=0. Let us
assume that this end segment is specified by s=N. We then obtain the constaint condition:
\begin{equation}
 z(N)=0
\end{equation}
The other end segment s=0 is free, and is called as the free end. The boundary condition at s=0 is given by:
\begin{equation}
 z'(0)=0,
\end{equation}
The maximum of polymer brush height is denoted as $h$, note that all coordinates are scaled by $h$. The normalized end point distribution function is $g(y)$.
The local tension function, $E(x,y)$, is defined as:
\begin{equation}
 E(x,y)=\frac{dz(s)}{ds}|^{z(0)=y}_{z(s)=x}, x<=y
\end{equation}
the inverse of $E(x,y)$, denoted as $\phi(x,y)$.
We can write down the monomer density $\Phi(x)$ as
\begin{equation}
 \Phi(x)=\int^{1}_{x} g(y)\phi(x,y)dy=\int^{1}_{x} g(y)E^{-1}(x,y)dy
 \label{eqn:dens}
 \end{equation}
the free energy of polymer brush melt mainly comes from the elascity of polymer chains, which can be writen as,
\begin{equation}
 f_{e}=3/4\Delta^2\int^{1}_{0}dy\int^{y}_{0}dxg(y)E(x,y),  \Delta=h/a\sqrt{N}
\end{equation}

Two constraints are needed, one is the incompressibility of polymer brush,i.e., 
\begin{equation}
 \Phi(x)=\int^{1}_{x} g(y)E^{-1}(x,y)dy=1
\end{equation}
and the equal length condition, which requires that polymer chain has fixed length of $N$.
\begin{equation}
 \Lambda(y)=\int^{y}_{0}dxE^{-1}(x,y)=1
 \label{eqn:equal-length}
\end{equation}
Now minimize the free energy subject to the two above constraints, 
we define:
\begin{equation}
 \Omega=f_e+\int^{1}_{0}dx\phi(x)\Phi(x)+\int^{1}_{0}dy\lambda(y)\Lambda(y)
\end{equation}
Set the variation of $\Omega$ with respect to $E(x,y)$, $g(y)$, we have:
\begin{eqnarray}
\frac{D\Omega}{DE} =\frac{3}{4}\Delta^2g(y)-[g(y)\phi(x)+\lambda(y)]E^{-2}(x,y)=0
\end{eqnarray}
\begin{eqnarray}
\frac{D\Omega}{Dg}=\int^{y}_{0}\frac{3}{4}\Delta^2E(x,y)+\phi(x)E^{-1}(x,y)=0
\label{eqn:F3}
\end{eqnarray}

Now we have the SSL nonlinear equation set as:
\begin{eqnarray}
F_1(\mathbf{X})= \Phi(x)-1=\int^{1}_{x} g(y)E^{-1}(x,y)dy-1=0 \\
F_2(\mathbf{X})=\Lambda(y)-1=\int^{y}_{0}dxE^{-1}(x,y)-1=0 \\
F_3(\mathbf{X})=\int^{y}_{0}\frac{3}{4}\Delta^2E(x,y)+\phi(x)E^{-1}(x,y)dx=0
\end{eqnarray}
where $\mathbf{X}=({g,\phi,\lambda})$, and
\begin{equation}
 E^{2}(x,y)=\frac{4}{3}\Delta^{-2}(\phi(x)+\frac{\lambda(y)}{g(y)})
 \label{eqn:Exy}
\end{equation}







\section{Classical theory sloution}
The Classical trajectory corresponds to the solution $z(s)$ that starts from $z(0)$ and arrives at $z(1)$ at the time $s=1$, irrespective
of its initial position $z(0)$. Such a feature is realized by adopting the isochonisms of a pendulum, thus we can the solution of
the trajectory as
\begin{equation}
 z(s)=z_0\cos(\pi s/2)
\end{equation}
now we can calculate the distribution function $\phi(x,y)$,
\begin{equation}
 \phi(x,y)=\frac{ds}{dz(s)}=\frac{2}{\pi}\frac{1}{\sqrt{y^2-x^2}}
\end{equation}
The monomer density of polymer brush is given by Eq.\ref{eqn:dens}, the $g(y)$ function can be inverted to obtain,
\begin{equation}
 g(y)=\frac{y}{(1-y^2)^{1/2}}
\end{equation}

Substitue the above sloution of $g(y)$ and $\phi(x,y)$ into the SSL theory,
we can prove that 
\begin{equation}
 \phi(x)=-\frac{3}{4}\Delta^{2}\frac{\pi^2 x^2}{4}
\end{equation}
thus inserting these solutions to Eq.\ref{eqn:Exy}, we have,
\begin{equation}
 \lambda(y)=\frac{3\pi^2}{16}\Delta^2y^2g(y)=\frac{3\pi^2}{16}\Delta^2\frac{y^3}{(1-y^2)^{1/2}}
\end{equation}

\section{Numerical implemention of SSL theory}
Since we have the solution from classical trajectory theory, we can use these solutions as the input for the nonlinear 
equations derived from SSL theory. The trick part of the numerical solution is to evalue the numerical integrals 
accurately, since there are singular functions like $g(y)$ and $E^{-1}(x,y)$ invloved. Some numerical experiments are done before
finally solve the problem.

We rely on the interpolation of the coarse grid when the integrals contains the singular part.
For example, to accurately evalue the integral of the normalized $g(y)$ function, i.e.,
$\int^{1}_{0}g(y)dy=1$, we need to carefully deal with the singular part when $y$ approaches 1.

We have two set of grids, one is coarse and the other is fine, the coarse grid has $Nx$ grid sites and is defined as
\begin{equation}
 x_c(i)=(i+1)\frac{1}{(Nx+1)}, i=0,1,2,...Nx-1
\end{equation},
and the fine grid is defined as
\begin{equation}
 x_f(i,j)=x_c(i-1)+(j+1)\frac{1}{(Nx+1)(Nm+1)}, j=0,1,2,...Nm-1
 \end{equation}
 for each $i$ defined on the coarse grid.
Note that $x_c(i=-1)=0$ and $x_c(i=Nx)=1$, although both are not included in our computation.

We set $Nx=100$ and $Nm=200$, when dealing with the integral of $g(y)$, first, we use the polynomial  interpolation for 
the last $N_f$ points on the coarse grid and get the value of $g(y)$ on the fine grid. The value of the integral is
the summation of the $Nx-N_f$ grids on the coarse grid and the rest part on the fine grid.
Below is some numerical results we tested.
\begin{center}
    \begin{tabular}{ l | l | l |}
    \hline
    Nx & Nm & Integral value \\ \hline
     100 & 200 & 0.948365880741 \\ \hline
     100 & 400 & 0.948502392298 \\ \hline
     200 & 200 & 0.962247412428 \\ \hline
     200 & 400 & 0.962346717998 \\ \hline
     400 & 200 & 0.971917239189 \\ \hline
    \end{tabular}
\end{center}

It seems that polynomial interpolation is not good, but still fine, however, the later numerical test will 
show that the polynomial interpolation should not be used when dealing with the sigular function such as 
$g(y)=\frac{y}{\sqrt(1-y^2)}$. The problem is that we not only need to do the interpolation, but also the extrapolation,
and it turns out that polynomial expansion is not suitable to extrapolate the singular function.

\begin{figure}[h]
\includegraphics[width=12cm]{gy2.eps}
\caption{The polynomial extrapolation at the range of $x\in[0.99,1.0]$. The rank of polynomial
expansion and the interval of the interpolated interval on the coarse grid are tuned to improve the
result. However, this is the best we can get.}
\end{figure}


Now we need to use a more suitable interpolation scheme to improve the accuracy of the numerical integral.
Below we suggest a numerical trick to avoid the sigular problem. for example,
\begin{equation}
 \Phi(0)=\int^{1}_{0}g(y)E^{-1}(x,y)dy
\end{equation}
instead of doing the interpolation with $g(y)$, we could change the integral as
\begin{equation}
 \Phi(0)=\int^{1}_{0}(g(y)\sqrt{1-y^2})\frac{1}{\sqrt{1-y^2}}E^{-1}(x,y)dy
\end{equation}

Note that the combination $g(y)\sqrt{1-y^2}$ is no longer singular at $y=1$, since we expect that $g(y)$ takes the form in the classical
trajectory, and we can do the polynomial interpolation for the combination function.
\begin{figure}[h]
\includegraphics[width=12cm]{gy3}
\caption{Attempt to eliminate the singularity at $y=1$, as explained above.}
\end{figure}

Using this scheme, we can increase the accuracy of numerical evaluation of $\Phi(0)$ from $0.70$ to $0.91$,
still not very good, but it is an improvement indeed.

\section{ Numerical details for solving the nonlinear equations from SSL theory}
One remaining nasty problem of solving the nonlinear equations set is that when the solution variables are
ajusted towards the converged solution, the $E^2(x,y)$ function, according to Eq.\ref{eqn:Exy}, is very likely to
become negative, which produces $nan$ and the nonlinear solver eventually fails to work.

We find another trick that might avoid the above difficulties. Since the $g(y)$, $E^{-1}(x,y)$ functions are singular,
(actually, there is a term for such singularity called Maxwellian singularity), we can replace the original functions
we aim to solve by:
\begin{equation}
 f(y)=\sqrt{1-y^2}g(y) 
 \end{equation}
 \begin{equation}
 G(x,y)=\sqrt{y^2-x^2}E^{-1}(x,y)
\end{equation}

Now we can do the integral of $g(y)$ as:
\begin{equation}
 \int^{1}_{0}\frac{f(y)}{\sqrt{1-y^2}}dy=\int^{\pi/2}_{0}f(\sin\theta)d\theta
\end{equation}
where $y=\sin\theta$, since we expect $g(y) \rightarrow \frac{y}{\sqrt{1-y^2}}$, $f(y)$ function shoud be a smooth function and the integral
should be done easily.

Similarly, we can do the integral of $E^{-1}(x,y)$ as:
\begin{equation}
\Lambda(y)= \int^{y}_{0}\frac{G(x,y)}{\sqrt{1-(\frac{x}{y})^2}}d(\frac{x}{y})=\int^{\pi/2}_{0}G(y\sin t,y)dt
\end{equation}
where $\sin t=\frac{x}{y}$, and with such replacement, we can do the integral of the smooth $G(x,y)$ function.
Eq.\ref{eqn:dens} can be now calculated as:
\begin{equation}
 \Phi(x)=\int^{1}_{x}g(y)E^{-1}(x,y)dy=\int^{\pi/2}_{\theta'}f(\sin \theta)\frac{G(x,\sin\theta)}{\sqrt{\sin^2\theta-x^2}}d\theta
\end{equation}
we substract the remaining singularity as:
\begin{equation}
 \int^{\pi/2}_{\theta'}(f(\sin \theta)\frac{G(x,\sin\theta)}{\sqrt{\sin^2\theta-x^2}}-\frac{\sin\theta}{\sqrt{\sin^2\theta-x^2}}d\theta
\end{equation}
and the remaining part can be integrated analytically,
\begin{equation}
 \int^{1}_{x}\frac{y/\sqrt{1-y^2}}{\sqrt{y^2-x^2}}dy=1
\end{equation}


And we rewrite Eq.\ref{eqn:Exy} as :
\begin{equation}
 G^{-2}(x,y)(y^2-x^2)=\frac{4}{3}\Delta^{-2}(\phi(x)+\lambda(y)\frac{\sqrt{1-y^2}}{f(y)})
\end{equation}

We will solve the SSL equations on the mesh of variables $\theta$ and $t$ mesh with uniform spacing:
\begin{equation}
 \theta_i=t_i=(i+1)h, \quad i=0,1,...,N_{\theta}-1, \qquad h=1/N_{\theta}
\end{equation}

and variable $x$ and $y$ are connected with $\theta$ and $t$ through:
\begin{equation}
 y_i=\sin \theta_i ; \qquad  x_i=y_i\sin t_i=\sin\theta_i\sin t_i
\end{equation}
the functions to be solved are now replaced by:
\begin{eqnarray}
\begin{align*}
 g(y)   \rightarrow  f(\theta) : f(\theta_i)=\sqrt{1-\sin^2(\theta_i)}g(\sin\theta_i) \\
  E(x,y) \rightarrow G(\theta,t) : G(\theta_i,t_j)=\sqrt{\sin^2(\theta_i)-\sin^2\theta_i\sin^2(t_i)}E(\sin\theta_i\sin t_j,\sin\theta_i) \\
  E(x,y) \rightarrow G'(\theta,t) : G'(\theta_i,t_j)=\sqrt{\sin^2(\theta_i)-\sin^2(t_i)}E(\sin t_j,\sin\theta_i) \qquad t_j<=\theta_i \\
  \phi(x) \rightarrow \phi(\theta_i) \\
  \lambda(y) \rightarrow \lambda(\theta_i) \\
 \end{align*}
\end{eqnarray}

the SSL nonlinear integral equations now become:
\begin{eqnarray}
 F_1(i)=\Phi(i)-1=\int^{\pi/2}_{t_i}f(\theta)\frac{G'(\theta,t_i)}{\sqrt{\sin^2(\theta)-\sin^2(t_i)}}d\theta-1\\
 F_2(i)=\Lambda(i)-1=\int^{\pi/2}_{0}G(\theta_i,t)dt-1\\
 F_3(i)=\int^{\pi/2}_{0}G(\theta_i,t)\phi(t)+\frac{3}{4}\Delta^2 \frac{(\sin^2\theta_i-\sin^2\theta_i\sin^2t)}{G(\theta_i,t)}dt
\end{eqnarray}
where 
\begin{eqnarray}
 G^{-2}(\theta_i,t_j)\sqrt{\sin^2\theta_i-\sin^2\theta_i\sin^2t_j}=\frac{3}{4}\Delta^2(\phi(\sin\theta_i\sin t_j) +\lambda(\sin\theta_i)\frac{\sqrt{1-\sin^2\theta_i}}{f(\sin\theta_i)})\\
 G'^{-2}(\theta_i,t_j)\sqrt{\sin^2\theta_i-\sin^2t_j}=\frac{3}{4}\Delta^2(\phi(\sin t_j) +\lambda(\sin\theta_i)\frac{\sqrt{1-\sin^2\theta_i}}{f(\sin\theta_i)}) \qquad t_j<=\theta_i\\
\end{eqnarray}
Here we must emphasize that $\textcolor{red}{G(\theta_i,t_j)}$ and $\textcolor{red}{G'(\theta_i,t_j)}$ are two different functions.
The solution functions are defined on evenly spaced $\theta$ or $t$ mesh grid, that is,
\begin{eqnarray}
 f_i=f(\sin\theta_i) \\
 \phi_i=\phi(\sin\theta_i)\\
 \lambda_i=\lambda(\sin\theta_i)\\
\end{eqnarray}
thus we need to provide the value of $\phi(\sin\theta_i\sin t_j)$ from the given collection of values of $\phi(\sin\theta_i)$.
\subsection{coding details}
First, we test the accuracy of numerical integral with the analytical solution,

%the simplest scheme is the  trapzoidal rule, such as:
%\begin{equation}
%\int^{t_i}_{0} f(\sin t)dt=h(\frac{1}{2}f(0)+\sum^{i-1}_{j=1}f_j +\frac{1}{2}f_i)
%\end{equation}








\section{corrections to the entropy of free end}
Now we account for the entropic contribution due to the free ends as
\begin{equation}
 f_g=\int^{1}_{0}dyg(y)ln(g(y)/\Delta)
\end{equation}
and this correction only leads to a modification of Eq.\ref{eqn:F3},
\begin{equation}
 \frac{D\Omega}{Dg}=\int^{y}_{0}\frac{3}{4}\Delta^2E(x,y)+\phi(x)E^{-1}(x,y)+ln(g(y)/\Delta)+1=0
\end{equation}







\end{document}
